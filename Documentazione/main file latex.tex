\documentclass[a4paper,12pt]{article}
\usepackage[T1]{fontenc}
\usepackage[utf8]{inputenc}
\usepackage[italian]{babel}
\usepackage{hyperref}
\begin{document}
\pagestyle{empty}	


\title{Relazione progetto sito web Centro Commerciale Tito Levi Civita}
\begin{minipage}[c][\textheight][c]{\textwidth}\centering
	\textsc{Universit\`a degli studi di Padova}
	\\ \noindent
	\textsc{Dipartimento di Matematica Pura e Applicata}
	\\ \noindent 
	\textsc{Corso di laurea in Informatica}
	\\ \noindent
	\textsc{Anno Accademico 2017-2018}
	\vfill
	\textbf{\large Progetto del corso di Tecnologie Web}
	\\ \hfill \\
	\textbf{\huge Centro commerciale\\ Tullio Levi Civita\\}
	\vfill
	\textsc{\Large a cura di:}
	\\ \noindent
	\textsc{\Large Giovanni Bergo:		1126144\\ Bianca Ciuche:		1122193 \\Daniel Rossi:		1125444\\ Manfredi Smaniotto:	1123057\\}	
\end{minipage}

\tableofcontents
\pagebreak
\section{Informazioni di accesso}
Il sito è accessibile al seguente link:
\begin{center}
	http://tecweb2016.studenti.math.unipd.it/darossi
\end{center}
Come primo account amministratore è possibile utilizzare:
\begin{itemize}
	\item username: admin
	\item password: admin
\end{itemize}
Vi sono anche alcuni account per utenti negozio come ad esempio:
\begin{itemize}
	\item username:
	\item password:
\end{itemize}
Dal pannello amministratore è possibile aggiungere account; non è concessa la connessione con due account contemporaneamente poichè le variabili di sessione vengono mantenute fino al logout successivo.
\section{Scopo del progetto e target}
\subsection{Scopo del progetto}
Il progetto si pone l'obbiettivo di creare un sito internet di un centro commerciale, dove devono essere disponibili le principali informazioni riguardanti orari di apertura, offerte, prodotti dei negozi, comunicazioni di servizio e le indicazioni per raggiungere il centro commerciale. \\
L'utente web può quindi ricevere informazioni come sopra elencato utilizzando sia piattaforme desktop che dispositivi mobile.\\
Il proprietario del negozio ha invece la possibilità di accedere ad una sezione dedicata nell'area riservata dove poter personalizzare il proprio sito modificando gli orari di apertura, inserendo offerte, prodotti, le immagini e le descrizioni da presentare.\\
L'utente amministratore può invece gestire gestire tutti gli account presenti all'interno della piattaforma e scrivere alcune comunicazioni da dare ai clienti riguardo al centro commerciale.\\
\subsection{Target di utenza}
Il target del sito risulta essere l'utenza visitatrice dei centri commerciali, quindi un pubblico molto vasto ed eterogeneo che richiede la creazione di un sito adatto a tutti e senza particolari adattamenti per una particolare tipologia di pubblico.\\
La parte privata è invece pensata appositamente per un utilizzatore con buone conoscenze informatiche e indirizzato ad un utilizzo della piattaforme prettamente da computer fisso (verrà comunque data una versione mobile di quella parte del sito).
\section{Organizzazione delle informazioni}
\subsection{Introduzione}
I contenuti informativi presenti all'interno del sito sono stati organizzati in modo tale da lasciare le informazioni di maggiore importanza come le comunicazioni di servizio, gli orari e le ultime promozioni nella home del sito, lasciando spazio nelle altre pagine ad una ricerca più dettagliata in base al desiderio di informazioni dell'utente.\\
La parte privata cerca di raggruppare, soprattutto per quanto riguarda la parte negozi, i form e le possibilità di modifica in base al tipo di informazioni che si desidera inserire (un esempio sono le pagine apposite per le promozionio o per i prodotti).\\
Il sito si compone delle pagine che quì di seguito andremo a descrivere.
\subsection{Parte pubblica}
La parte pubblica è tutto quell'insieme di pagine liberamente accessibili all'interno del sito senza la necessità di alcun account. La composizione di tale parte è data da:
\begin{itemize}
	\item \textbf{home (index.php)}: contiene le informazioni di base, in particolare sulla sinistra vi sono tutte le ultime promozioni annunciate per i vari negozi, mentre sulla destra troviamo gli orari di apertura e chiusura del centro commerciale e i principali avvisi di servizio.\\
	
	\item \textbf{negozi (negozi.php)}: contiene la lista completa dei negozi presenti all'iterno del centro.\\
	Cliccando su un logo di un negozio è possibile accedere alla pagina dedicata al particolare negozio.
	\begin{itemize}
		\item \textbf{pagina negozio ()}: pagina dedicata al negozio in cui sono riportati gli orari di apertura, i prodotti e le promozioni del dato negozio.\\
		Tramite click su un prodotto o su una promozione è possibile accedere alle relative informazioni nella pagina dedicata.
		\begin{itemize}
		\item \textbf{pagina prodotto (prodpromo.php)} pagina contenente l'immagine del prodotto associato ad una descrizione.\\
		\item \textbf{pagina promozione (prodpromo.php)}: pagina contenente l'immagine della promozione associata ad una descrizione e alle date di inizio e fine di essa.
	\end{itemize}
	\item \textbf{dove siamo (dovesiamo.php)}: pagina contenente le informazioni per raggiungere il centro commerciale
	\item \textbf{contatti (contatti.php)}: pagina contenente le principali informazioni riguardanti i contatti utilizzabili per comunicare con il centro commerciale
	\item \textbf{pagina promozioni (promozioni.php)}: pagina creata per contenere tutte le promozioni dei vari negozi.
	\end{itemize}
\end{itemize}
\subsection{Parte privata}
Insieme di pagine accessibili con credenziali per admin o per addetti dei negozi per la modifica dei relativi contenuti.
\subsubsection{Admin}
\begin{itemize}
	\item \textbf{pagina generale (generale\_admin.php)}: pagina dedicata alla gestione degli account dei negozi (creazione ed eliminazione) oltre che alla modifica della password amministratore.
	\item \textbf{pagina eventi (eventi\_private.php)}: pagina creata per la gestione dei messaggi di servizio del centro commerciale, in particolare con essa si possono comunicare eventi come aperture/chiusure straordinarie del centro oppure delle novità; viene anche data la possibilità di cancellazione di tali messaggi nel caso venissero fatti degli errori di digitazione.
\end{itemize}
\subsubsection{Negozi}
\begin{itemize}
	\item \textbf{pagina generale (generale\_private.php)}: pagina principale della sezione privata dedicata ai negozi in cui sono compresi la modifica di varie informazioni principali (contatti, logo, nome negozio, orario negozio e descrizione) e la modifica della password.
	\item \textbf{pagina promozioni (promozioni\_private.php)}: pagina dedicata all'inserimento e rimozione delle promozioni.\\
	Cliccando su una promozione si viene reindirizzati ad una pagina per la modifica della promozione.
	\begin{itemize}
		\item \textbf{pagina modifica promozione (promo\_mod.php)}: pagina dedicata alla modifica di ogni campo dato di una singola promozione 
	\end{itemize}

	\item \textbf{pagina prodotti (prodotti\_private.php)}: pagina dedicata all'inserimento e rimozione dei prodotti.\\
	Cliccando su un prodotto si viene reindirizzati ad una pagina per la modifica del prodotto.
	\begin{itemize}
		\item \textbf{pagina modifica prodotto (prod\_mod.php)}: pagina dedicata alla modifica di ogni campo dato di un singolo prodotto
	\end{itemize}
	
\end{itemize}
\section{Usabilità}
\subsection{Introduzione}
Il progetto è stato creato perseguendo per quanto possibile la semplicità d'uso e l'intuitività nell'utilizzo del sito. In particolare si è cercato di utilizzare alcuni espedienti per facilitare al publico meno avvezzo alla tecnologia la fruizione di ogni sezione del sito.
\subsection{Grafica e layout}
\subsubsection{Layout adattivo}
La larghezza della pagina e i relativi contenuti sono scalati mano a mano che viene ridotta l'ampiezza dello schermo. La massima larghezza testata per il sito è stata di 1920x1080px, risoluzione massima con cui la maggior parte delle persone fruiscono i siti internet in questo momento.
\subsection{Animazioni}
Nel progetto sono state inserite alcune animazioni per rendere più appaganti o più chiare alcune azioni all'iterno del sito.
\begin{itemize}
	\item \textbf{Page loading}: Si è scelto di fare utilizzo di un'animazione di caricamento poichè, anche se l'utenza generalmente non ama questo tipo di soluzioni, ci sembrava poco sensato lasciare visibile la pagina durante il suo caricamento. Per evitare quindi di presentare agli utenti la pagina ancora in caricamento l'utilizzo di questa animazione ci è parso essere l'unico percorso davvero sensato da seguire.
	
	\item \textbf{Segnalazioni errori form parte privata}: l'inserimento di informazioni nei form è sempre controllato tramite php, comportando la stampa di una conferma di inserimento dei dati se questa ha successo, oppure una segnalazione errori che indichi subito all'utente le correzioni da apporre.
\end{itemize}
\subsection{Gestione errore 404}
Il gruppo intendeva personalizzare la gestione di tale pagina di errore reindirizzando l'utente su una pagina apposita ma non vi è stato modo di accedere al file .htaccess per poter compiere tale passaggio. La pagina di errore di ricerca della pagina quindi è rimasta quella di default.
\subsection{Ottimizzazione per dispositivi mobile}
La navigazione da cellulare è una feature da non trascurare per un sito di un centro commerciale, poichè la ricerca dell'informazione all'ultimo momento come gli orari dei negozi o gli ultimi sconti possono essere informazioni vitali per alcuni utenti del sito. Si sono quindi ridotte le immagini presenti sulla versione mobile del sito e si è cercato di portare il menu del sito sul fondo della pagina in modo da poter essere acceduto tramite un pulsante in cima alla pagina, senza richiedere l'utilizzo di javascript per delle animazioni non sempre accessibili da tutti i browser.
\subsection{Menù orizzontale}
Anche se esso risulta essere una scelta poco rivolta ad una successiva estensione del sito si è scelto di utilizzare un menù di questo tipo rispetto ad un menù laterale poichè il numero di link presenti su di esso non risultava così elevato da richiedere più spazio in estensione orizzontale rispetto a quanto reso disponibile da uno schermo di dimensioni anche ridotte.\\
Per migliorarne l'usabilità si è voluto sottolineare il link dove è stata acceduta la pagina, disattivandone il link sul menù nella parte alta, mentre il menù nella parte bassa comprende un'ancora sul link della pagina attuale per poter essere riportati in cima ad essa.
\section{Accessibilità}
\subsection{Introduzione}
Nella progettazione del sito si è voluto controllare fin da subito che ogni singolo elemento inserito non compromettesse l'accessibilità del sito. Sono state seguite le linee guida WCAG per quanto possibile e si è testata la loro applicazione tramite gli strumenti di test dell'accessibilità che verranno successivamente citati.
\subsection{Accorgimenti adottati}
\begin{itemize}
	\item \textbf{JavaScript}: il sito prevede un utilizzo di JavaScript prettamente accessorio e non necessario alla fruizione del sito; ogni singolo controllo o animazione risulta essere non necessario alla navigazione o viene replicato normalmente tramite la stampa di messaggi simili con chiamate php.\\
	In particolare JavaScript viene utilizzato per il controllo degli input nei form e per le animazioni di caricamento delle pagine.
	\item \textbf{Breadcrumb}: ogni pagina presenta una barra che informa l'utente della pagina che attualmente stà visitando e quale percorso ha seguito per arrivarci.
	\item \textbf{Immagini e link nascosti}: oltre a inserire le alternative testuali delle immagini 
\end{itemize}
\section{Note sullo sviluppo}
\subsection{Tecnologie impiegate}
Le tecnologie utilizzate per la realizzazione del sito sono:
\begin{itemize}
	\item XHTML
	\item CSS3
	\item JavaScript
	\item PHP v7.1
\end{itemize}
\subsection{struttura parte amministrazione}
\subsection{Fogli di stile}
Il sito utilizza quattro fogli di stile dedicati per la parte utente (dimensione normale(style.css), tablet(tablet.css), mobile(mobile.css), stampa(print.css) e tre fogli di stile per la parte amministratore (dimensione normale(private\_style.css), tablet(private\_tablet.css), mobile(private\_mobile.css)
\subsection{Validazione dei Form}
La parte di amministrazione e inserimento contenuti del sito è ricca di form, quindi si avranno avvisi sia in javascript sia in html riguardanti gli errori trovati nei campi di inserimento durante la procedura di invio dei dati; gli avvisi in php avverranno rispetto alla parte javascript nella parte alta della pagina, indicando la tipologia di errore rilevata o l'avvenuto inserimento dei dati in modo corretto.\\
Le funzioni di validazione dei form sono create ad hoc, poichè essendo esse molto poco omogenee come richieste non danno la possibilità di raggruppare in un'unica funzione tutti i controlli da svolgere.\\
L'assenza di javascript non è problematica per il controllo di tali form, poichè gli appositi controlli vengono svolti quantomeno da una funzione server-side in php prima di salvare i dati nel database.\\
\subsection{Dimensionamento immagini}
Le immagini inserite vengono scalate appositamente affinchè non vi siano problemi nella loro visualizzazione all'interno del sito. Tale funzionalità è stata inserita per evitare di dover vincolare ad una dimensione fissata l'inserimento delle immagini nel sito. Nell'eventualità in cui le immagini non venissero scalate nel modo desiderato è sempre possibile sostituirle con un'altra immagine correttamente dimensionata prima dell'inserimento nel sito.

\section{Compatibilità}
Durante lo sviluppo del sito è stato chiaro sin dal primo momento che fosse necessario supportare un'ampia gamma di browser per accedere al sito.
\\I browser che sono stati testati e controllati per il corretto funzionamento del sito sono:
\begin{itemize}
	\item Chrome - Versione
	\item Mozilla Firefox - Versione
	\item Opera - Versione
	\item Microsoft Edge - Versione
	\item Internet Explorer - Versione
\end{itemize}

\section{Test e validazione}
\subsection{Introduzione}
Questa sezione è dedicata alla sintesi dei test effettuati e alla descrizione degli strumenti utilizzati a tale scopo.\\
Preso atto che gli strumenti automatici non sono in senso assoluto un indicatore sufficiente per definire un sito accessibile, essi sono un requisito necessario per poter affermare che il nostro sito sia quantomeno conforme alle linee guida consigliate.\\
Poichè il sito permette il caricamento di contenuti il risultato di alcuni test, con particolare riferimento ai colori, potrebbe non essere lo stesso se eseguito una volta popolato il sito. L'esito di questi test è quindi valido soltanto nelle parti esenti dal caricamento di immagini degli utenti.
\subsection{Strumenti utilizzati}
\begin{itemize}
	\item \textbf{W3C Validator}: servizio di validazione del markup disponibile al link:\\
	\url{https://validator.w3.org/};
	
	\item \textbf{Total Validator}:validatore di accessibilità e correttezza del markup disponibile in versione gratuita al sito:\\
	\url{http://www.totalvalidator.com/validator/Validator};
	
	\item \textbf{NVDA}: screen reader reperibile gratuitamente al link:\\
	\url{https://www.nvaccess.org/};
	
	\item \textbf{IE Accessibility Toolbar}: barra di strumenti per test di accessibilità compatibile solo con internet Explorer ed installabile al sito:\\
	\url{https://developer.paciellogroup.com/resources/wat/};
	
	\item \textbf{Cynthia Says}: validatore di accessibilità disponibile al sito:\\
	\url{http://www.contentquality.com};
	
	\item \textbf{Toptal Colorblind Web Page Filter}: simulatore di daltonismo utilizzato per visualizzare come il sito viene visto da un individuo affetto da Protanopia, Deutanopia, Tritanopia e Achromatopsia, disponibile gratuitamente al sito:\\
	\url{https://www.toptal.com/designers/colorfilter};
	
	\item \textbf{WebAIM Color Contrast Checker}: validatore di contrasto del colore disponibile gratuitamente al sito:\\
	\url{https://webaim.org/resources/contrastchecker/};
\end{itemize}
\subsection{Esiti}
\begin{itemize}
	\item \textbf{W3C Validator}: Tutte le pagine sono state validate e non risulta che esse abbiano errori di validazione del codice.
	
	\item \textbf{Total Validator}:validatore di accessibilità e correttezza del markup disponibile in versione gratuita al sito:\\
	\url{http://www.totalvalidator.com/validator/Validator};
	
	\item \textbf{NVDA}: screen reader reperibile gratuitamente al link:\\
	\url{https://www.nvaccess.org/};
	
	\item \textbf{IE Accessibility Toolbar}: barra di strumenti per test di accessibilità compatibile solo con internet Explorer ed installabile al sito:\\
	\url{https://developer.paciellogroup.com/resources/wat/};
	
	\item \textbf{Cynthia Says}: validatore di accessibilità disponibile al sito:\\
	\url{http://www.contentquality.com};
	
	\item \textbf{Toptal Colorblind Web Page Filter}: simulatore di daltonismo utilizzato per visualizzare come il sito viene visto da un individuo affetto da Protanopia, Deutanopia, Tritanopia e Achromatopsia, disponibile gratuitamente al sito:\\
	\url{https://www.toptal.com/designers/colorfilter};
	
	\item \textbf{WebAIM Color Contrast Checker}: validatore di contrasto del colore disponibile gratuitamente al sito:\\
	\url{https://webaim.org/resources/contrastchecker/};
\end{itemize}

\end{document}