\documentclass[a4paper,11pt]{article}
\usepackage[T1]{fontenc}
\usepackage[italian]{babel}
\begin{document}
\pagestyle{empty}	


\title{Relazione progetto sito web Centro Commerciale Tito Levi Civita}
\begin{minipage}[c][\textheight][c]{\textwidth}\centering
	\textsc{Universit\`a degli studi di Padova}
	\\ \noindent
	\textsc{Dipartimento di Matematica Pura e Applicata}
	\\ \noindent
	\textsc{Corso di laurea in Informatica}
	\\ \noindent
	\textsc{Anno Accademico 2017-2018}
	\vfill
	\textbf{\large Progetto del corso di Tecnologie Web}
	\\ \noindent
	\textbf{\huge Centro commerciale}
	\\ \noindent
	\textbf{\Huge Tullio Levi Civita}
	\vfill
	\textsc{\Large a cura di:}
	\\ \noindent
	\textsc{\Large Giovanni Bergo, Bianca Ciuche, Daniel Rossi, Manfredi Smaniotto}	
\end{minipage}

\tableofcontents
\pagebreak
\section{Abstract}
Il progetto si pone l'obbiettivo di creare un sito internet di un centro commerciale in cui sono presenti 3 (?) negozi. L'utente web deve quindi poter ricevere informazioni di contatto e posizione del centro commerciale, cercare rapidamente offerte e accedere rapidamente al sito del negozio.
\\ \noindent Il sito è stato pensato affinchè vengano mostrate nella home e nelle pagine non riferite ai negozi le news e tutte le informazioni riguardanti la struttura nel suo complesso. I negozi invece avranno le loro sezioni dedicate per lasciare informazioni ai propri clienti o reindirizzarli ai rispettivi siti internet personalizzati, lasciando quindi la responsabilità ai gestori dei locali di gestire i rispettivi spazi web personalizzati. Verrà comunque concesso uno spazio sulla home gestito dai detentori del sito del centro commerciale per eventuali comunicazioni su offerte o notizie particolari riguardo ai negozi.
\\ \noindent

\section{Analisi dell'utenza del sito}

\section{Progettazione}
	
\end{document}